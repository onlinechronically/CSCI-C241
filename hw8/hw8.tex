\documentclass{article}
\usepackage{graphicx}
\usepackage{amsthm}
\usepackage{amsmath}
\usepackage{amssymb}
\usepackage{geometry}
\usepackage{tikz}
\usetikzlibrary{arrows}

\geometry{a4paper, total={170mm,257mm}, left=20mm, top=20mm}
\AtBeginEnvironment{align}{\setcounter{equation}{0}} 

\title{HW8 (CSCI-C241)}
\author{Lillie Donato}
\date{19 March 2024}

\begin{document}

\maketitle

\begin{enumerate}
    \item Question One
    \begin{enumerate}
        \item $R_1$ is not a function, because $R(1, 2)$ and $R(1, 3)$ but $2 \neq 3$
        \item $R_2$ is a total function each member of $A$ is realated only to a single member of $A$
        \item $R_3$ is a partial function, because each member of $A$ is related to no more than one member of $A$, but not every member of $A$ is related to another member.
        \item $P_2$ is partial function because if $x,y,z \in \mathbb{R}$, $x \cdot y=120$, and $x \cdot z = 120$, then $x = z$, but $\neg P_2(0, n)$ where $n \in \mathbb{R}$
        \item $P_3$ is a total function because if $x,y,z \in \mathbb{R}^*$, $x \cdot y=120$, and $x \cdot z = 120$, then $x = z$, and for every $n \in \mathbb{R}^*$, there is some $n_2 \in \mathbb{R}^*$ where $n*n_2 = 120$
    \end{enumerate}
    \item Question Two
    \begin{enumerate}
        \item $f_1$ is not one-to-one, because $f_1(a) = f_1(d)$ but $a \neq d$
        \item $f_1$ is not onto because there is no $x \in B$ where $f_1(x) = a$
        \item $f_2$ is one-to-one because there is no $x,y \in B$ where $f_2(x) = f_2(y)$ and $x \neq y$
        \item $f_2$ is onto because for every member of the codomain, there is some member of the domain that maps to said member of the codomain 
        \item $s_1$ is not one-to-one because $s_1(10) = s_1(-10)$ but $10 \neq -10$
        \item $s_2$ is one-to-one because if a $x \in [0,\infty)$ and $x^2 - 10 = n$ where $n \in [0, \infty)$, there is no other member of the domain that can be squared and have ten added to equal $n$
        \item $s_2$ is not onto, because there is no $x \in [0,\infty)$ where $s_2(x) = 1$
        \item $c_1$ is one-to-one because there is no $x,y \in \mathbb{R}$ where $x \neq y$ and $c_1(x) = c_1(y)$
        \item $c_1$ is onto because for every $y \in \mathbb{R}$ there will always be some $x \in \mathbb{R}$ where $y = x^3 -10$
        \item $c_2$ is not onto necause there is no $x \in \mathbb{Z}$ where $c_2(x) = 1$
        \item $d$ is a total function, because for any possible string (including the empty string), there exists a string that contains dashes in between each character
        \item $d$ is one-to-one, because for any two strings, if they have dashes inserted in between their characters, the resulting strings for each would never be the same
        \item $d$ is not onto, because there exists no $s \in \text{Str}$ where $d(s) = \text{wall}$
        \item $f$ is a partial function, because for any $s \in \text{Str}$, there is only one possible first character of that string, but there does not exist $s_2 \in \text{Str}$ such that $f(\text{""}) = s_2$
        \item $f$ is not one-to-one, because $f(\text{car}) = f(\text{can})$ but $\text{car} \neq \text{can}$
    \end{enumerate}
    \item Question Three
    \begin{enumerate}
        \item Part (a)
        \item Part (b)
        \item Part (c)
        \item Part (d)
        \item Part (e)
        \item Part (f)
        \item Part (g)
        \item Part (h)
        \item Part (i)
        \item Part (j)
        \item Part (k)
        \item Part (l)
        \item Part (m)
        \item Part (n)
    \end{enumerate}
    \item Question Four
    \begin{enumerate}
        \item Part (a)
        \item Part (b)
        \begin{proof}
            \begin{align}
                &
            \end{align}
        \end{proof}
    \end{enumerate}
    \item Question Five
    \begin{enumerate}
        \item Part (a)
        \item Part (b)
    \end{enumerate}
    \item Question Six
    \begin{enumerate}
        \item Part (a)
        \item Part (b)
        \begin{proof}
            \begin{align}
                &
            \end{align}
        \end{proof}
    \end{enumerate}
    \item Question Seven
    \begin{enumerate}
        \item Part (a)
        \begin{proof}
            \begin{align}
                &
            \end{align}
        \end{proof}
    \end{enumerate}
\end{enumerate}

\end{document}