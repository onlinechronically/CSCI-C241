\documentclass{article}
\usepackage{graphicx}
\usepackage{amsthm}
\usepackage{amsmath}
\usepackage{amssymb}
\usepackage{geometry}
\usepackage{tikz}
\usetikzlibrary{arrows}

\geometry{a4paper, total={170mm,257mm}, left=20mm, top=20mm}
\AtBeginEnvironment{align}{\setcounter{equation}{0}} 

\title{HW8 (CSCI-C241)}
\author{Lillie Donato}
\date{19 March 2024}

\begin{document}

\maketitle

\begin{enumerate}
    \item Question One
    \begin{enumerate}
        \item $R_1$ is not a function, because $R(1, 2)$ and $R(1, 3)$ but $2 \neq 3$
        \item $R_2$ is a total function each member of $A$ is realated only to a single member of $A$
        \item $R_3$ is a partial function, because each member of $A$ is related to no more than one member of $A$, but not every member of $A$ is related to another member.
        \item $P_2$ is partial function because if $x,y,z \in \mathbb{R}$, $x \cdot y=120$, and $x \cdot z = 120$, then $x = z$, but $\neg P_2(0, n)$ where $n \in \mathbb{R}$
        \item $P_3$ is a total function because if $x,y,z \in \mathbb{R}^*$, $x \cdot y=120$, and $x \cdot z = 120$, then $x = z$, and for every $n \in \mathbb{R}^*$, there is some $n_2 \in \mathbb{R}^*$ where $n*n_2 = 120$
    \end{enumerate}
    \item Question Two
    \begin{enumerate}
        \item $f_1$
        \item $f_1$
        \item $f_2$
        \item $f_2$
        \item $s_1$
        \item $s_2$
        \item $s_2$
        \item $c_1$
        \item $c_1$
        \item $c_2$
        \item $d$
        \item $d$
        \item $d$
        \item $f$
        \item $f$
    \end{enumerate}
    \item Question Three
    \begin{enumerate}
        \item Part (a)
        \item Part (b)
        \item Part (c)
        \item Part (d)
        \item Part (e)
        \item Part (f)
        \item Part (g)
        \item Part (h)
        \item Part (i)
        \item Part (j)
        \item Part (k)
        \item Part (l)
        \item Part (m)
        \item Part (n)
    \end{enumerate}
    \item Question Four
    \begin{enumerate}
        \item Part (a)
        \item Part (b)
        \begin{proof}
            \begin{align}
                &
            \end{align}
        \end{proof}
    \end{enumerate}
    \item Question Five
    \begin{enumerate}
        \item Part (a)
        \item Part (b)
    \end{enumerate}
    \item Question Six
    \begin{enumerate}
        \item Part (a)
        \item Part (b)
        \begin{proof}
            \begin{align}
                &
            \end{align}
        \end{proof}
    \end{enumerate}
    \item Question Seven
    \begin{enumerate}
        \item Part (a)
        \begin{proof}
            \begin{align}
                &
            \end{align}
        \end{proof}
    \end{enumerate}
\end{enumerate}

\end{document}