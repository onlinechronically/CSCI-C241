\documentclass{article}
\usepackage{graphicx}
\usepackage{amsthm}
\usepackage{amsmath}
\usepackage{amssymb}
\usepackage{geometry}
\usepackage{tikz}
\usetikzlibrary{arrows}

\geometry{a4paper, total={170mm,257mm}, left=20mm, top=20mm}
\AtBeginEnvironment{align}{\setcounter{equation}{0}} 

\title{HW8 (CSCI-C241)}
\author{Lillie Donato}
\date{19 March 2024}

\begin{document}

\maketitle

\begin{enumerate}
    \item Question One
    \begin{enumerate}
        \item $R_1$ is not a function, because $R(1, 2)$ and $R(1, 3)$ but $2 \neq 3$
        \item $R_2$ is a totalfunction, because each member of $A$ is related to a single member of $A$
        \item $R_3$ is a partial function, because each member of $B$ is not related to more than a single member of $B$
        \item $P_2$ is a partial function, because it has uniqueness but $P$ ($0*y \neq 120$)
        \item $P_2$ is a total function (zero not included), because if a number multiplied by another is equal to $120$, then the first number can not be multiplied by any other number to equal $120$
    \end{enumerate}
    \item Question Two
    \begin{enumerate}
        \item $f_1$ is not one-to-one, because $f_1(a) = f_1(d)$ but $a \neq d$
        \item $f_1$ is not onto, because $a \in B$ but no members of $B$ relate to $a$
        \item $f_2$ is one-to-one, because every member of $B$ is related to only one member of $B$
        \item $f_2$ is onto, because every member of $B$ has at least one member of $B$ related to itself
        \item Part (e)
        \item Part (f)
        \item Part (g)
        \item Part (h)
        \item Part (i)
        \item Part (j)
        \item Part (k)
        \item Part (l)
        \item Part (m)
        \item Part (n)
        \item Part (o)
    \end{enumerate}
    \item Question Three
    \begin{enumerate}
        \item Part (a)
        \item Part (b)
        \item Part (c)
        \item Part (d)
        \item Part (e)
        \item Part (f)
        \item Part (g)
        \item Part (h)
        \item Part (i)
        \item Part (j)
        \item Part (k)
        \item Part (l)
        \item Part (m)
        \item Part (n)
    \end{enumerate}
    \item Question Four
    \begin{enumerate}
        \item Part (a)
        \begin{proof}
            \begin{align}
                &
            \end{align}
        \end{proof}
        \item Part (b)
        \begin{proof}
            \begin{align}
                &
            \end{align}
        \end{proof}
    \end{enumerate}
    \item Question Five
    \begin{enumerate}
        \item Part (a)
        \begin{proof}
            \begin{align}
                &
            \end{align}
        \end{proof}
        \item Part (b)
        \begin{proof}
            \begin{align}
                &
            \end{align}
        \end{proof}
    \end{enumerate}
    \item Question Six
    \begin{enumerate}
        \item Part (a)
        \begin{proof}
            \begin{align}
                &
            \end{align}
        \end{proof}
        \item Part (b)
        \begin{proof}
            \begin{align}
                &
            \end{align}
        \end{proof}
    \end{enumerate}
    \item Question Seven
    \begin{enumerate}
        \item Part (a)
    \end{enumerate}
\end{enumerate}

\end{document}