\documentclass{article}
\usepackage{graphicx}
\usepackage{amsthm}
\usepackage{amsmath}
\usepackage{amssymb}
\usepackage{geometry}
\usepackage{tikz}
\usetikzlibrary{arrows}

\geometry{a4paper, total={170mm,257mm}, left=20mm, top=20mm}
\AtBeginEnvironment{align}{\setcounter{equation}{0}} 

\title{HW8 (CSCI-C241)}
\author{Lillie Donato}
\date{19 March 2024}

\begin{document}

\maketitle

\begin{enumerate}
    \item Question One
    \begin{enumerate}
        \item $R_1$ is not a function, because $R(1, 2)$ and $R(1, 3)$ but $2 \neq 3$
        \item $R_2$ is a total function each member of $A$ is realated only to a single member of $A$
        \item $R_3$ is a partial function, because each member of $A$ is related to no more than one member of $A$, but not every member of $A$ is related to another member.
        \item $P_2$ is partial function because if $x,y,z \in \mathbb{R}$, $x \cdot y=120$, and $x \cdot z = 120$, then $x = z$, but $\neg P_2(0, n)$ where $n \in \mathbb{R}$
        \item $P_3$ is a total function because if $x,y,z \in \mathbb{R}^*$, $x \cdot y=120$, and $x \cdot z = 120$, then $x = z$, and for every $n \in \mathbb{R}^*$, there is some $n_2 \in \mathbb{R}^*$ where $n*n_2 = 120$
    \end{enumerate}
    \item Question Two
    \begin{enumerate}
        \item $f_1$ is not one-to-one, because $f_1(a) = f_1(d)$ but $a \neq d$
        \item $f_1$ is not onto because there is no $x \in B$ where $f_1(x) = a$
        \item $f_2$ is one-to-one because there is no $x,y \in B$ where $f_2(x) = f_2(y)$ and $x \neq y$
        \item $f_2$ is onto because for every member of the codomain, there is some member of the domain that maps to said member of the codomain 
        \item $s_1$ is not one-to-one because $s_1(10) = s_1(-10)$ but $10 \neq -10$
        \item $s_2$ is one-to-one because if a $x \in [0,\infty)$ and $x^2 - 10 = n$ where $n \in [0, \infty)$, there is no other member of the domain that can be squared and have ten added to equal $n$
        \item $s_2$ is not onto, because there is no $x \in [0,\infty)$ where $s_2(x) = 1$
        \item $c_1$ is one-to-one because there is no $x,y \in \mathbb{R}$ where $x \neq y$ and $c_1(x) = c_1(y)$
        \item $c_1$ is onto because for every $y \in \mathbb{R}$ there will always be some $x \in \mathbb{R}$ where $y = x^3 -10$
        \item $c_2$ is not onto necause there is no $x \in \mathbb{Z}$ where $c_2(x) = 1$
        \item $d$ is a total function, because for any possible string (including the empty string), there exists a string that contains dashes in between each character
        \item $d$ is one-to-one, because for any two strings, if they have dashes inserted in between their characters, the resulting strings for each would never be the same
        \item $d$ is not onto, because there exists no $s \in \text{Str}$ where $d(s) = \text{wall}$
        \item $f$ is a partial function, because for any $s \in \text{Str}$, there is only one possible first character of that string, but there does not exist $s_2 \in \text{Str}$ such that $f(\text{""}) = s_2$
        \item $f$ is not one-to-one, because $f(\text{car}) = f(\text{can})$ but $\text{car} \neq \text{can}$
    \end{enumerate}
    \item Question Three
    \begin{enumerate}
        \item $\{(1,a),(2,a),(3,b)\}$
        \item This is not possible, because in order for a relation to be a function, each member of the domain can't be related to more than a single member of the codomain.
        \item $\{(a,2),(b,3),(c,1)\}$
        \item $f(x) = 2x$
        \item $f(x) = 2x$
        \item $f(x) = \lfloor \frac{x}{10} \rfloor$
        \item $f(x) = x+5$
        \item $f(s) = s \text{ is concatenated to itself}$
        \item $f(s) = \begin{cases}
                \texttt{""} &\text{ if } s = \texttt{""} \\
                s \text{ with the last character removed } &\text{ otherwise}
            \end{cases}$
        \item $f(s) = |s|$
        \item $|\texttt{Str}| \geq |\mathbb{N}|$
        \item $f(s) = \text{the sum of the ascii values of each character of } s \text{ multiplied by the total amount of all ascii values,}$ \\
        raised to the position (starting at zero) of said character in $s$ \\
        Note: This is base-256 to base-10, similar to how we convert base-16 to base-10
        \item $|\texttt{Str}| = |\mathbb{N}|$
    \end{enumerate}
    \item Question Four
    \begin{enumerate}
        \item $k$ is not one-to-one, because $0, 6 \in \mathbb{R}$ and $k(0) = k(6)$, but $0 \neq 6$
        \item Claim: $k_2$ is one-to-one
        \begin{proof}
            \begin{align}
                &\text{Choose } x_1,x_2 \in (3, \infty) \text{ and Assume } k_2(x_1) = k_2(x_2) \\
                &\hspace{1cm} \text{Since } k_2(x_1) = k_2(x_2), \text{ we know } (x_1 - 3)^2 = (x_2 - 3)^2 \\
                &\hspace{1cm} \text{Suppose towards a contradiction } x_1 \neq x_2 \\
                &\hspace{2cm} \text{Since } (x_1 - 3)^2 = (x_2 - 3)^2 \text{, we know } \pm (x_1 - 3) = \pm (x_2 - 3) \\
                &\hspace{2cm} \text{Case 1: } (x_1 - 3) = (x_2 - 3) \\
                &\hspace{3cm} \text{Since } (x_1 - 3) = (x_2 - 3) \text{, we know } x_1 = x_2 \\
                &\hspace{2cm} \text{In the case of $(x_1 - 3) = (x_2 - 3)$, we proved $x_1 = x_2$,} \\
                &\hspace{2cm} \text{which contradicts our assumption} \nonumber \\
                &\hspace{2cm} \text{Case 2: } (x_1 - 3) = - (x_2 - 3) \\
                &\hspace{3cm} \text{Since } (x_1 - 3) = - (x_2 - 3) \text{, we know } x_1 - 3 = - x_2 + 3 \\
                &\hspace{3cm} \text{Since } x_1 - 3 = - x_2 + 3 \text{, we know } x_1 - 6 = - x_2 \\
                &\hspace{3cm} \text{Since } x_1 - 6 = - x_2 \text{ and $x_2 > 3$, we know } x_1 < 3 \\
                &\hspace{2cm} \text{In the case of $(x_1 - 3) = - (x_2 - 3)$, we proved an impossibility of $x_1 < 3$,} \\
                &\hspace{2cm} \text{which contradicts our domain} \nonumber \\
                &\hspace{2cm} \text{Case 3: } - (x_1 - 3) = (x_2 - 3) \\
                &\hspace{3cm} \text{Since } - (x_1 - 3) = (x_2 - 3) \text{, we know } - x_1 + 3 = x_2 - 3 \\
                &\hspace{3cm} \text{Since } - x_1 + 3 = x_2 - 3 \text{, we know } - x_1 = x_2 - 6 \\
                &\hspace{3cm} \text{Since } - x_1 = x_2 - 6 \text{ and $x_1 > 3$, we know } x_2 < 3 \\
                &\hspace{2cm} \text{In the case of $- (x_1 - 3) = (x_2 - 3)$, we proved an impossibility of $x_2 < 3$,} \\
                &\hspace{2cm} \text{which contradicts our domain} \nonumber \\
                &\hspace{2cm} \text{Case 4: } - (x_1 - 3) = - (x_2 - 3) \\
                &\hspace{3cm} \text{Since } - (x_1 - 3) = - (x_2 - 3) \text{, we know } - x_1 + 3 = - x_2 + 3 \\
                &\hspace{3cm} \text{Since } - x_1 + 3 = - x_2 + 3 \text{, we know } - x_1 = - x_2 \\
                &\hspace{3cm} \text{Since } - x_1 = - x_2 \text{, we know } x_1 = x_2 \\
                &\hspace{2cm} \text{In the case of $- (x_1 - 3) = - (x_2 - 3)$, we proved $x_1 = x_2$,} \\
                &\hspace{2cm} \text{which contradicts our assumption} \nonumber \\
                &\hspace{2cm} \text{In any case of } (x_1 - 3)^2 = (x_2 - 3)^2 \text{, we proved an impossibility of } x_1 \neq x_2 \\
                &\hspace{1cm} \text{Under the assumption of } x_1 \neq x_2 \text{, we proved an impossibility, so } x_1 = x_2 \\
                &\text{Under the assumption of } k_2(x_1) = k_2(x_2) \text{, we proved } x_1 = x_2 \text{, so $k_2$ is one-to-one}
            \end{align}
        \end{proof}
    \end{enumerate}
    \item Question Five
    \begin{enumerate}
        \item $h$ is not one-to-one, because $\texttt{"X-Y-Z"}, \texttt{"X--Y--Z"} \in \texttt{Str}$ and $h(\texttt{"X-Y-Z"}) = h(\texttt{"X--Y--Z"})$, but $\texttt{"X-Y-Z"} \neq \texttt{"X--Y--Z"}$
        \item $h$ is not onto, because $\texttt{"-"} \in \texttt{Str}$ but there does not exist a $s \in \texttt{Str}$ where $h(s) = \texttt{"-"}$
    \end{enumerate}
    \item Question Six
    \begin{enumerate}
        \item $a$ is not one-to-one, because $\texttt{"76"},\texttt{"67"} \in \texttt{Str}$ and $a(\texttt{"76"}) = a(\texttt{"67"})$, but $\texttt{"76"} \neq \texttt{"67"}$
        \item Claim: $a$ is onto
        \begin{proof}
            \begin{align}
                &\text{Choose } n \in \mathbb{N} \\
                &\hspace{1cm} \text{Let } s \text{ be an empty string} \\
                &\hspace{1cm} \text{As long as $n > 9$, concatenate $\texttt{"9"}$ to the end of $s$ and decrease $n$ by 9}  \\
                &\hspace{1cm} n \text{ is concatenated to the end of } s \\
                &\text{For every } x \in \mathbb{N} \text{, there is some string containing digits that add up to } x
            \end{align}
        \end{proof}
    \end{enumerate}
    \item Question Seven
    \begin{enumerate}
        \item Claim: $|A| \neq |B|$
        \begin{proof}
            \begin{align}
                &\text{Choose the sets } A,B \\
                &\hspace{1cm} \text{Let $F$ be a function from $A \rightarrow B$ and $G$ be a function from $B \rightarrow A$}
            \end{align}
        \end{proof}
    \end{enumerate}
\end{enumerate}

\end{document}