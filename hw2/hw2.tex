\documentclass{article}
\usepackage{graphicx} % Required for inserting images

\title{HW2 (CSCI-C241)}
\author{Lillie Donato}
\date{23 January 2024}

\begin{document}

\maketitle

\begin{enumerate}
    \item Question One
    \begin{enumerate}
        \item The pair ($A \equiv A \lor A$) is logically equivalent as shown in the truth table below:
        
        \begin{tabular}{| c | c |}
            \hline
            A & $A \lor A$ \\
            \hline
            true & true \\
            \hline
            false & false \\
            \hline
        \end{tabular}
        \item The pair ($A \equiv A \oplus A$) is not logically equivalent as shown in the truth assignment below:
        
        A = true
        
        $A \equiv A \oplus A$

        $true \equiv true \oplus true$

        $true \equiv false$
        \item The pair ($A \rightarrow B \equiv \neg A \rightarrow \neg B$) is not logically equivalent as shown in the truth assignment below:
        
        A = true
        
        B = false

        $A \rightarrow B \equiv \neg A \rightarrow \neg B$

        $true \rightarrow false \equiv false \rightarrow true$

        $false \equiv true$
        % 
        \item The pair ($P \leftrightarrow \neg Q \equiv (P \land \neg Q) \lor (\neg P \land Q)$) is logically equivalent as shown in the truth table below:
        
        \begin{tabular}{| c | c | c | c |}
            \hline
            P & Q & $P \leftrightarrow \neg Q$ & $(P \land \neg Q) \lor (\neg P \land Q)$ \\
            \hline
            true & true & false & false \\
            \hline
            true & false & true & true \\
            \hline
            false & true & true & true \\
            \hline
            false & false & false & false \\
            \hline
        \end{tabular}
        \item
        The pair ($A \lor B \equiv P \lor Q$) is not logically equivalent as shown in the truth assignment below:

        A = true

        B = true

        P = false

        Q = false

        $A \lor B \equiv P \lor Q$

        $true \lor true \equiv false \lor false$
        
        $true \equiv false$
        \item The pair ($A \lor \neg A \equiv P \rightarrow P$) is logically equivalent as shown in the truth table below:
        
        \begin{tabular}{| c | c | c | c |}
            \hline
            A & P & $A \lor \neg A$ & $P \rightarrow P$ \\
            \hline
            true & true & true & true \\
            \hline
            true & false & true & true \\
            \hline
            false & true & true & true \\
            \hline
            false & false & true & true \\
            \hline
        \end{tabular}
    \end{enumerate}
    \item Question Two
    \begin{enumerate}
        \item
        The argument is not logically equivalent as shown in the truth assignment below:

        A = true

        B = false

        $A \rightarrow B$ = $true \rightarrow false$ = false

        $B \rightarrow A$ = $false \rightarrow true$ = true

        $A \rightarrow B \equiv B \rightarrow A$ = $false \equiv true$ = false
        \item
        The argument is logically equivalent as shows in the truth table below:

        \begin{tabular}{| c | c | c | c |}
            \hline
            A & B & $A \rightarrow B$ & $\neg B \rightarrow \neg A$ \\
            \hline
            true & true & true & true \\
            \hline
            true & false & false & false \\
            \hline
            false & true & true & true \\
            \hline
            false & false & true & true \\
            \hline
        \end{tabular}
    \end{enumerate}
    \item Question Three
    \begin{enumerate}
        \item
        This argument is not valid because in the truth assignment below, the premise is true but the conclusion is not:
        
        A = false

        B = false

        Premise = $\neg (A \land B) \land \neg A$ = $\neg (false \land false) \land \neg false$ = $true \land true$ = true

        Conclusion = $\neg (B \rightarrow A)$ = $\neg (false \rightarrow false)$ = $\neg true$ = false
        \item This argument is valid because in the truth table below, each premise that is true, has a corresponding conclusion that is also true:
        
        \begin{tabular}{| c | c | c | c | c |}
            \hline
            X & Y & $Y \rightarrow X$ & $X \rightarrow Y$ & $\neg Y \land X$ \\
            \hline
            true & true & true & true & true \\
            \hline
            true & false & true & false & true \\
            \hline
            false & true & false & true & false \\
            \hline
            false & false & true & true & true \\
            \hline
        \end{tabular}
        \item
        This argument is not valid because in the truth assignment below, the premise is true but the conclusion is not:
        
        P = false

        Q = false

        Premise = $(P \rightarrow \neg Q) \land (\neg Q)$ = $(false \rightarrow true) \land (\neg false)$ = $true \land true$ = true

        Conclusion = $\neg \neg P$ = $\neg \neg false$ = false
        \item
        This argument is not valid because in the truth assignment below, the premise is true but the conclusion is not:
        
        P = true

        Q = true

        R = false

        Premise = $P \land Q$ = true

        Conclusion = $R$ = false
    \end{enumerate}
    \item Question Four
    \begin{enumerate}
        \item $A \oplus B$
        \item $A \lor \neg A$
        \item This is not possible, because for a formula to be a contingency it must be both satisfiable and not satisfiable, where a contradiction must be only not satisfiable.
        \item $(A \land B) \lor (\neg A \land \neg B)$
        \item This is not possible, because for a formula to be a tautology it must be satisfiable for every assignment, where a contingency must be both satisfiable and not satisfiable.
        \item \{ $P \land \neg Q$, $P$, $\neg Q$, $\neg (\neg P \lor Q)$ \}
        \item \{ $P \lor \neg Q$, $\neg P \land Q$ \}
        \item This is not possible, because for a formula to be consistent there must be one assignment where all of the formula's in the set are satisfiable at once, but for a contradiction there must be no satisfiable assignments.
        \item \{ $(A \lor B) \lor (\neg A \lor \neg B)$, $A \oplus B$ \}
        \item 
        \begin{tabular}{c}
            $P$ \\
            $Q$ \\
            \hline
            $\neg P \land \neg Q$ \\
        \end{tabular}
        \item $p = X \lor Y$ and $q = \neg (X \oplus Y)$
        \item $p = P \lor Q$ and $q = \neg P \oplus \neg Q$
    \end{enumerate}
    \item Question Five
    \begin{enumerate}
        \item Yes, this is because for a formula to be a tautology means every assignment must be satisfiable, therefore meaning that the formula is satisfiable.
        \item This is not true, because in order for a formula to be a tautology, each and every assignment must be satisfiable, where for a formula to be satisfiable it only requires one assignment.
        \item Yes, this is the case because in order for a formula to be a contingency, it must be not only satisfiable but also not satisfiable for a different assignment.
    \end{enumerate}
    \item Question Six - all letters true due to that being the only way one can be true
    \begin{enumerate}
        \item The formula is valid, because in order for the premise to evaluate to true, both $A$ and $B$ must be true, in turn meaning $C$ also must be true. When looking at the last statement in the premise, if $C$ is true so must $D$ \& $E$. With that being said, if all the letters are true so would the conclusion.
    \end{enumerate}
    \item Question Seven
    \begin{enumerate}
        \item $(A \land B) \lor (C \rightarrow (D \land E))$ would have a larger truth table due to the truth table for this formula having a greater number of different atomic propositions. Furthermore, the equation to calculate the amount of rows in a truth table, is $2^n$ where n is the number of different atomic propositions.
    \end{enumerate}
\end{enumerate}

\end{document}
