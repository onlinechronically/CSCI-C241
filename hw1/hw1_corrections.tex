\documentclass{article}
\usepackage{graphicx} % Required for inserting images

\title{HW1 (Corrections) (CSCI-C241)}
\author{Lillie Donato}
\date{16 January 2024}

\begin{document}

\maketitle

% \section{Introduction}
\begin{itemize}
    \item Question Two
    \begin{itemize}
        \item 2f

        P = Participants were timed on this task

        Q = Most finished in less than 8 minutes

        $\neg P \land Q$
        \item 2j
        
        P = The people will give up their arms

        Q = The tyrant resigns

        R = We get our money back

        $P \rightarrow (Q \land R)$
        \item 2k
        
        P = The Lyapunov function exists

        Q = The system is stable

        $P \leftrightarrow Q$
        \item 2l
        
        P = The Turing Test was passed

        Q = The individal is intelligent

        $P \rightarrow Q$
        \item 2m
        
        P = There are no antibodies in the subject’s body

        Q = The subject is susceptible to an infection

        $P \rightarrow \neg Q$
        \item 2n
        
        P = Disciplinary knowledge is used

        Q = Organizational skills is used

        R = The teaching is considered effective

        $R \rightarrow (P \land Q)$
    \end{itemize}
    \item Question Five
    \begin{itemize}
        \item 5c
        
        The statement is not a contradiction as shown in the following truth assignment:
        
        X = false

        $X \rightarrow \neg X$ = $false \rightarrow \neg false$ = $false \rightarrow true$ = $true$
        \item 5d
        
        The statement is not a tautology as shown in the following truth assignment:
        
        A = false

        B = true

        $\neg A \rightarrow \neg (A \lor B)$ = $\neg true \rightarrow \neg (true \lor false)$ = $false \rightarrow true$ = $false$
        \item 5e
        
        The statement is a contingency as shown in the following truth assignments:
        
        \begin{enumerate}
            \item
            A = true

            B = true

            $\neg A \rightarrow \neg (A \lor B)$ = $\neg true \rightarrow \neg (true \lor true)$ = $false \rightarrow false$ = $true$

            \item
            A = false

            B = true

            $\neg A \rightarrow \neg (A \lor B)$ = $\neg false \rightarrow \neg (false \lor true)$ = $true \rightarrow false$ = $false$
        \end{enumerate}
        \item 5g
        
        The statement is satisfiable as shown in the following truth assignment:
        
        A = true

        B = true

        C = true
        
        $((A \rightarrow B) \land (C \lor \neg B)) \rightarrow (A \rightarrow C) = ((true \rightarrow true) \land (true \lor false)) \rightarrow (true \rightarrow true)$ = $(true \land true) \rightarrow true$ = $true \rightarrow true$ = $true$
        \item 5i
        
        The statement is not a contradiction as shown in the following truth assignment:
        
        A = true

        B = true

        $(A \rightarrow B) \rightarrow (\neg A \rightarrow \neg B) = (true \rightarrow true) = true$
        \item 5j
        
        The statement is not a tautology as shown in the following truth assignment:
        
        A = false

        B = true

        $(A \rightarrow B) \rightarrow (\neg A \rightarrow \neg B) = (false \rightarrow true) \rightarrow (true \rightarrow false) = true \rightarrow false = false$
        \item 5k
        
        The statement is satisfiable as shown in the following truth assignment:

        A = false

        B = true

        C = false

        D = true

        $\neg A \lor ((D \lor \neg D) \rightarrow ((B \land \neg B) \leftrightarrow (C \rightarrow C))) = \neg false \lor ((true \lor \neg true) \rightarrow ((true \land \neg true) \leftrightarrow (false \rightarrow false))) = true \lor (true \rightarrow (false \leftrightarrow true)) = true$
    \end{itemize}
\end{itemize}

\end{document}
