\documentclass{article}
\usepackage{graphicx}
\usepackage{amsthm}
\usepackage{amsmath}
\usepackage{amssymb}
\usepackage{geometry}
\usepackage{tikz}
\usetikzlibrary{arrows}

\geometry{a4paper, total={170mm,257mm}, left=20mm, top=20mm}
\AtBeginEnvironment{align}{\setcounter{equation}{0}}
\AtBeginEnvironment{eqnarray}{\setcounter{equation}{0}}

\title{HW10 (CSCI-C241)}
\author{Lillie Donato}
\date{DD April 2024}

\begin{document}

\maketitle

\begin{enumerate}
    \item Question One
    \begin{enumerate}
        \item $1^3 + 2^3 + 3^3 = 1 + 8 + 27 = 36$
        \item $\frac{1}{3} + \frac{1}{4} = \frac{7}{12}$
        \item $\sqrt{2}$
        \item $\frac{1}{1} = 1$
        \item $\frac{1}{k+1}$
        \item $\frac{1}{k}$
        \item $\sum\limits_{i=1}^{k}{\frac{1}{i}} + \frac{1}{k+1}$
    \end{enumerate}
    \item Question Two
    \begin{enumerate}
        \item Claim: For $n \in \mathbb{N}$, $n \geq 1$, $\sum\limits_{i=1}^{n}{2^{i-1} = 2^n - 1}$
        \begin{proof}
            (induction on $n$) \\
            (Base Step, $n=1$):
            \begin{eqnarray}
                \sum\limits_{i=1}^{n}{2^{1-1}} &=& \sum\limits_{i=1}^{n}{2^0} \\
                &=& \sum\limits_{i=1}^{n}{1} \\
                &=& 1 \\
                &=& 2^0 \\
                &=& 2^{1-1}
            \end{eqnarray}
            (Inductive Step): \\
            Assume $\sum\limits_{i=1}^{k}{2^{i-1}} = 2^k - 1$ for some $k \geq 1$
            \begin{eqnarray}
                \sum\limits_{i=1}^{k+1}{2^{i-1}} &=& \sum\limits_{i=1}^{k}{2^{i-1}} + 2^{k+1-1} \\
                &=& \sum\limits_{i=1}^{k}{2^{i-1}} + 2^k \\
                &=& 2^k - 1 + 2^k \hspace{1cm} \text{(by the induction hypothesis, and $2^k = 2^k$)} \\
                &=& 2^k + 2^k - 1 \\
                &=& 2 \cdot 2^k - 1 \\
                &=& 2^{k+1} - 1
            \end{eqnarray}
        \end{proof}
        \item Claim: For all $n \in \mathbb{N}$, $\sum\limits_{i=0}^{n}{i! \cdot i} = (n+1)! - 1$
        \begin{proof}
            (induction on $n$) \\
            (Base Step, $n=0$):
            \begin{eqnarray}
                \sum\limits_{i=0}^{n}{0! \cdot 0} &=& \sum\limits_{i=0}^{n}{0} \\
                &=& 0 \\
                &=& 1 - 1 \\
                &=& 1! - 1 \\
                &=& (0+1)! - 1
            \end{eqnarray}
            (Inductive Step): \\
            Assume $\sum\limits_{i=0}^{k}{i! \cdot i} = (k+1)! - 1$ for some $k \in \mathbb{N}$
            \begin{eqnarray}
                \sum\limits_{i=0}^{k+1}{i! \cdot i} &=& (\sum\limits_{i=0}^{k}{i! \cdot i}) + ((k+1)! \cdot (k+1)) \\
                &=& (k+1)! - 1 + (k+1)! \cdot (k+1) \hspace{1cm} \text{(By the induction hypothesis)} \\
                &=& (k+1)! + (k+1)! \cdot (k+1) - 1 \\
                &=& (k+1)! \cdot (1 + (k+1)) - 1 \\
                &=& (k+1)! \cdot (k+2) - 1 \\
                &=& (k+2)! - 1 \\
                &=& (k+1+1)! - 1
            \end{eqnarray}
        \end{proof}
        \item Claim: For $n \in \mathbb{N}$, $n \geq \text{the sum of the number of digits in } n$
        \begin{proof}
            (induction on the number of digits of $n$) \\
            (Base Step, $n$ has one digit):
            \begin{eqnarray}
                &&\text{Since } n \text{ has one digit } n < 10 \\
                &&\text{Since } n<10 \text{ and the number of digits of } n=1, \text{ the sum of the digits is } n
            \end{eqnarray}
            (Inductive Step):
        \end{proof}
    \end{enumerate}
    \item Question Three
    \begin{enumerate}
        \item The minimum value where $f(x) = g(x)$ is $10^{12}$
        \item A value where $f(x) > g(x)$ is $19$
    \end{enumerate}
    \item Claim: For every positive real number $a$ where $a \geq e$, there exists $m \in \mathbb{N}$ such that for all $n \geq m$, $n! > a^n$
    \begin{proof}
        \begin{align}
            &\text{Choose } a \in \mathbb{R} \text{ such that }  a \geq e \\
            &\text{Suppose } m \in \mathbb{N} \text{ such that } n \geq m \\
            &\hspace{1cm} \text{Since half of the numbers that are being multiplied in } n! \text{ are greater than } \frac{n}{2} \text{,} \\
            &\hspace{1cm} \text{we know } \frac{n}{2} \text{ numbers of } n! \text{ are greater than } \frac{n}{2} \nonumber \\
            &\hspace{1cm} \text{Since } \frac{n}{2} \text{ numbers are greater than } \frac{n}{2} \text{, we know } n! > \left (\frac{n}{2}\right )^{\frac{n}{2}} \\
            &\hspace{1cm} \text{Since } n! > \frac{n}{2} \text{, we know } n! > \left (\frac{n}{2} \right )^{\frac{n}{2}} \\
            &\hspace{1cm} \text{Since } n \geq m \text{, we know } \left (\frac{n}{2}\right )^{\frac{n}{2}} \geq \left (\frac{m}{2}\right )^{\frac{n}{2}} \\
            &\hspace{1cm} \text{Since } \left (\frac{n}{2}\right )^{\frac{n}{2}} \geq \left (\frac{m}{2}\right )^{\frac{n}{2}} \text{, we know } n! > \left (\frac{m}{2}\right )^{\frac{n}{2}} \\
            &\hspace{1cm} \text{Since } \left (\frac{m}{2}\right )^{\frac{n}{2}} = \left (\sqrt{\frac{m}{2}}\right )^{n} \text{, we know } n! > \left (\sqrt{\frac{m}{2}}\right )^{n} \\
            &\hspace{1cm} \text{Let } m > 2a^2 \\
            &\hspace{1cm} \text{Since } m > 2a^2 \text{, we know } \frac{m}{2} > a^2 \\
            &\hspace{1cm} \text{Since } \frac{m}{2} > a^2 \text{, we know } \sqrt{\frac{m}{2}} > a \\
            &\hspace{1cm} \text{Since } \sqrt{\frac{m}{2}} > a \text{, we know } \left (\sqrt{\frac{m}{2}}\right )^n > a^n \\
            &\hspace{1cm} \text{Since } \left (\sqrt{\frac{m}{2}}\right )^n > a^n \text{, we know } n! > a^n
        \end{align}
    \end{proof}
    \item Question Five - Combinatorics
    \begin{enumerate}
        \item $10! = 3628800$
        \item $\frac{50!}{(50-5)!} = 254251200$
        \item $50^5 = 312500000$
        \item $\frac{20!}{16!} = 116280$
        \item $\frac{\frac{20!}{15!}}{5!} = 15504$
    \end{enumerate}
    \item Question Six
    \begin{enumerate}
        \item $52! = 8.0658 \times 10^{67}$ ($68$ digits)
        \item I personally do not think every permutation of a 52 deck card has been used. Despite how often cards are used in western culture and for how long they have been, I really do not believe it would be possible for $52!$ permutations to have been used because of the gigantic number it is.
    \end{enumerate}
    \item Claim: For any non-empty set $A$ of size $n$ and any integer $r$ with $n \geq r \geq 1$, there are $\frac{n!}{(n-r)!}$ permutations of length $r$ using values taken from $A$
    \begin{proof}
        (induction on $n$) \\
        (Base Step, $r=1$):
        \begin{eqnarray}
            \frac{n!}{(n-1)!} &=& \frac{n \cdot (n-1)!}{(n-1)!} \\
            &=& \frac{n}{1} \\
            &=& n
        \end{eqnarray}
        (Inductive Step): \\
        Assume there are $\frac{n!}{(n-k)!}$ permutations for some length $k \geq 1$ using values taken from $A$, a set of length $n$
        \begin{eqnarray}
            \frac{n!}{(n-(k+1))!} &=& \frac{n!}{(n-k-1)!} \\
            &=& \frac{n!}{\frac{(n-k) \cdot (n-k-1)!}{(n-k)}} \\
            &=& \frac{n!}{(n-k)!} \cdot (n-k) \\
        \end{eqnarray}
        Since there are $\frac{n!}{(n-k)!}$ permutations for some length $k$ (by the Induction Hypothesis), there are $n-k$ possibilities to create a new permutation of length $k+1$ from every permutation of length $k$, therefore there are $\frac{n!}{(n-k)!} \cdot (n-k)$ permutations for length $k+1$
    \end{proof}
\end{enumerate}

\end{document}