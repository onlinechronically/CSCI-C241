\documentclass{article}
\usepackage{graphicx}
\usepackage{amsthm}
\usepackage{amsmath}
\usepackage{amssymb}
\usepackage{geometry}
\usepackage{tikz}
\usetikzlibrary{arrows}

\geometry{a4paper, total={170mm,257mm}, left=20mm, top=20mm}
\AtBeginEnvironment{align}{\setcounter{equation}{0}}
\AtBeginEnvironment{eqnarray}{\setcounter{equation}{0}}

\title{HW11 (CSCI-C241)}
\author{Lillie Donato}
\date{16 April 2024}

\begin{document}

\maketitle

\begin{enumerate}
    \item Claim: For any graph $G$, if every pair of vertices in $G$ has one and only one path between them, then $G$ is a tree.
    \begin{proof}
        \begin{align}
            &\text{Choose a graph } G \text{ and assume every pair of vertices in } G \text{ has a single path between them} \\
            &\hspace{1cm} \text{Since every pair of vertices in } G \text{ has a single path between them, then we can conclude} \\
            &\hspace{1.5cm} \text{that every pair is connected} \nonumber \\
            &\hspace{1cm} \text{Since every pair of vertices in } G \text{ is connected, we can conclude } G \text{ is connected} \\
            &\hspace{1cm} \text{Suppose towards a contradiction that } G \text{ is not a tree} \\
            &\hspace{2cm} \text{Since } G \text{ is not a tree and } G \text{ is connected, we can conclude } G \text{ has a cycle} \\
            &\hspace{2cm} \text{Since } G \text{ has a cycle, we can conclude that there is some path } \\
            &\hspace{2.5cm} P = (V, a_1, a_2, \dots, a_{n-1}, a_n, W, b_1, b_2, \dots, b_{n-1}, b_n, V) \text{ for some pair of vertices } (V,W), \nonumber \\
            &\hspace{2.5cm} \text{where all vertices in the path } (a_1, a_2, \dots, a_{n-1}, a_n) \text{ are different from the vertices} \nonumber \\
            &\hspace{2.5cm} \text{in the path } (b_1, b_2, \dots, b_{n-1}, b_n) \\
            &\hspace{2cm} \text{Therefore, there are two distinct paths from $V$ to $W$ which contradicts our} \\
            &\hspace{2.5cm} \text{assumption that every pair of vertices has one and only one path between them} \\
            &\hspace{1cm} \text{Under the assumption that } G \text{ is not a tree, we proved an impossibility of there being} \\
            &\hspace{1.5cm} 2 \text{ distinct paths from one vertex to another, so } G \text{ must be a tree}
        \end{align}
    \end{proof}
    \item Claim: For any tree $t$, if you remove one edge from the tree, the resulting graph will not be connected.
    $\bold{Define} \colon \texttt{nd}(t) = \text{ the number of nodes in } t$
    \begin{proof}
        (induction on $\texttt{nd}(t)$) \\
        (Base Step, $\texttt{nd}(t) = 2$):
        \begin{align}
            &\text{Since } \texttt{nd}(t) = 2 \text{, there are two nodes in some tree } t \text{, meaning there is some edge } E \text{ that connects} \\
            &\hspace{0.5cm} \text{2 nodes } V \text{ and } W \\
            &\text{If } E \text{ is removed from } t \text{, then } V \text{ and } W \text{ are no longer connected, meaning } t \text{ is no longer connected}
        \end{align}
        (Inductive Step): \\
        Assume for some tree with $k$ nodes, if one edge is removed, then the resulting graph will no longer be connected
        \begin{align}
            &\text{Let } t = \text{ some tree with } k \text{ nodes} \\
            &\text{Let } t_1 = t \text{ with some node } n \text{ added to it, connected by some edge } e \text{, meaning it has } n+1 \text{ nodes} \\
            &\text{If } e \text{ is removed, } n \text{ is no longer connected to any other node in } t_1 \text{, meaning } t_1 \text{ is not connected} \\
            &\text{If any other edge in } t_1 \text{ is removed, it must be an edge in } t \text{, meaning by our Induction Hypothesis} \\
            &\hspace{0.5cm} t_1 \text{ will no longer be connected}
        \end{align}
    \end{proof}
    \item Question Three
    \begin{enumerate}
        \item $\verb|ed|(T) = \begin{cases}
            0 &\text{if } T = \verb|[]| \\
            \verb|ed|(T_1) + \verb|ed|(T_2) + 2 &\text{if } T = \verb|[|T_1, T_2\verb|]| \text{ for FBT's } T_1 \text{ and } T_2
        \end{cases}$
        \item Claim: For full every binary tree $t$, $\verb|nd|(t) = \verb|ed|(t)+1$ \\
        \begin{proof}
            \text{} \\
            (Base Step: $t = \texttt{[]}$)
            \begin{eqnarray}
                \texttt{nd}(t) &=& 1 \\
                &=& 0 + 1 \hspace{1cm} \text{(Because a tree with a single node has no children and no edges)}\\
                &=& \texttt{ed}(t) + 1 \\
            \end{eqnarray}
            (Inductive Step): \\
            Assume for some FBT's $T_1$ and $T_2$, $\texttt{nd}(T_1) = \texttt{ed}(T_1) + 1$ and $\texttt{nd}(T_2) = \texttt{ed}(T_2) + 1$
            \begin{align}
                &\text{Let } T = \texttt{[} T_1, T_2 \texttt{]} \\
                &\text{By the recursive definition of } \texttt{nd} \text{ and } \texttt{ed} \text{, we know } \texttt{nd}(T) = \texttt{nd}(T_1) + \texttt{nd}(T_2) + 1 \text{ and} \\
                &\hspace{0.5cm} \texttt{ed}(T) = \texttt{ed}(T_1) + \texttt{ed}(T_2) + 2 \\
                &\texttt{nd}(T) = \texttt{nd}(T_1) + \texttt{nd}(T_2) + 1 = \texttt{ed}(T_1) + 1 + \texttt{ed}(T_2) + 1 + 1 \hspace{1cm} \hspace{1cm} \text{(By the IH)} \\
                &\texttt{ed}(T_1) + 1 + \texttt{ed}(T_2) + 1 + 1 = \texttt{ed}(T_1) + \texttt{ed}(T_2) + 2 + 1 = \texttt{ed}(T) + 1
            \end{align}
        \end{proof}
        \item $log_2n$
        \item $2^d$
        \item $log_cn$
        \item $c^d$
        \item Claim: Prove $c^d$ \\
        $\bold{Define} \colon \texttt{dp}(t) = \text{ the depth of } t$
        \begin{proof}
            (induction on $\texttt{dp}(t)$) \\
            (Base Step, $\texttt{dp}(t) = 0$)
            \begin{eqnarray}
                c^0 &=& 1 \hspace{1cm} \text{(If the depth of a tree is $0$, then it must have a single node, its only leaf)}
            \end{eqnarray}
            (Inductive Step): \\
            Assume for some tree $t$ with at most $c$ children and a depth $k$, the largest possible number of leaves are $c^k$
            \begin{align}
                &\text{Since } t \text{ has at most } c^k \text{ leaves, a tree with a depth of } k+1 \text{ and at most } c \text{ children,} \\
                &\hspace{0.5cm} \text{would have } c \text{ children for every leaf, a total of } c^k \cdot c \text{ leaves} \\
                &c^k \cdot c = c^{k+1}
            \end{align}
        \end{proof}
    \end{enumerate}
\end{enumerate}

\end{document}