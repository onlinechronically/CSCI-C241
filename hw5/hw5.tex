\documentclass{article}
\usepackage{graphicx}
\usepackage{amsthm}
\usepackage{amsmath}
\usepackage{amssymb}

\title{HW5 (CSCI-C241)}
\author{Lillie Donato}
\date{6 February 2024}

\usepackage{geometry}
\geometry{a4paper, total={170mm,257mm}, left=20mm, top=20mm}

\begin{document}

\maketitle

\begin{enumerate}
    \item Question One
    \begin{enumerate}
        \item 2
        \item 1, 2, 6
        \item 1
        \item 6, 9
        \item $\frac{3}{2}$, 4, 2.5
        \item 1, 3, 4, 5
        \item 3, 4, 5, 6
        \item 1, 3
    \end{enumerate}
    \item Question Two
    \begin{enumerate}
        \item $\frac{1}{2}$
        \item -1
        \item 2
        \item 1, 8, 81
        \item $\{1\}$
        \item This is not possible, as there are no members in the empty set.
    \end{enumerate}
    \item Question Three
    \begin{enumerate}
        \item False
        \item True
        \item False
        \item False
        \item True
        \item False
        \item True
    \end{enumerate}
    \item Question Four
    \begin{enumerate}
        \item This would be true, as $a = 5, b = 6$, $5 \in A, 6 \in B$, and $5 + 6 = 11$
        \item This would be false, because the largest numbers in each set equal to $13$ ($a = 5, b = 8$)
        \item These are not equal, $1 \in D$, $1 \notin S$
        \item These are equal, as $C$ has the exact same members as $\{3, 5, 1\}$
        \item These are equal, as $C$ has the exact same members as $\{1, 5, 1, 3, 1, 5, 5, 1, 3\}$, even though the given set has duplicates those should be ignored as duplicates don't exist in sets
        \item These are not equal, $\varnothing \in \{\varnothing\}$, $\varnothing \notin \varnothing$
    \end{enumerate}
    \item Question Five
    \begin{enumerate}
        \item $\{1, 3, 5, 7, 9, 10\}$
        \item $\{4\}$
        \item $\{4\}$
        \item $\{0, 1, 2, 3, 4, 5, 6, 7, 8, 9, 10\}$
        \item $\{\}$
        \item $\{1, 3\}$
        \item $\{1, 3, 5\}$
        \item $\{1, 2, 3, 4, 5\}$
        \item $\{2, 4\}$
        \item $\{\}$
        \item $\{\}$
        \item $\{0, 2, 4, 6, 8\}$
    \end{enumerate}
    \item Question Six \\
    $\{\frac{x}{2} \mid x \in \mathbb{N} \land x \leq 8\}$
    \item Question Seven \\
    $\{2x \mid x \in \mathbb{N} \land x \leq 4\}$
    \item Question Eight
    \begin{enumerate}
        \item $| B | = 5$
        \item $| S | = 4$
        \item $| X | = 6$
        \item $| \{x \mid x \in \mathbb{N} \land x \leq 1000\} | = 1001$
        \item $| \varnothing | = 0$
        \item $| A \setminus B | = 3$
        \item $| Q | =$ infinite
        \item $| \mathbb{Z} | =$ infinite
    \end{enumerate}
    \item Question Nine
    \begin{enumerate}
        \item $\{\text{'a'}, \text{'b'}, \text{'c'}\}$
        \item $\{\text{'four'}, \text{'pink'}, \text{'roof'}\}$
        \item $\{\varnothing, \{\text{'a'}\}\}$
        \item $\{s \mid s \in \text{Str} \land | s | \text{ is even} \land | s | > 2\}$
        \item $\{\text{'a'}, \text{'b'}, \text{'c'}, \text{'d'}, \text{'e'}\}$
        \item $\{\text{'e'}, \text{'i'}\}$
        \item $\{\text{'jumper'}, \text{'public'}\}$
        \item $\{\{\text{'b'}, \text{'c'}\}\}$
    \end{enumerate}
    \item Question Ten
    \begin{enumerate}
        \item True
        \item True
        \item False
        \item True
        \item False
        \item True
        \item False
        \item False
        \item False
        \item False
        \item False
        \item True
        \item True
        \item False
        \item False
        \item True
        \item True
        \item True
    \end{enumerate}
    \item Question Eleven
    \begin{enumerate}
        \item False, $\text{i} \in V$, $\text{i} \notin A$
        \item True, because each string with a cardinality of four has an even cardinality
        \item False, $\text{fraction} \in S_{\text{even}}$, $\text{fraction} \notin S_4$
        \item False, because a powerset is a set of subsets, where the string 'a' by itself is not in any set.
        \item True, because a powerset is a set of subsets of set $A$, and both 'a' and 'b' are members of set $A$ 
        \item True, because a powerset is a set of subsets of set $A$, and 'a' is a member of set $A$
        \item False, $\{a, c\} \in X$, $\{a, c\} \notin Y$
    \end{enumerate}
    \item Question Twelve
    \begin{enumerate}
        \item $| \mathcal{P} (C)| = 2^{| C |} = 2^{3} = 8$
        \item $| \mathcal{P} (V)| = 2^{| V |} = 2^{5} = 32$
    \end{enumerate}
\end{enumerate}

\end{document}