\documentclass{article}
\usepackage{graphicx}
\usepackage{amsthm}
\usepackage{amsmath}
\usepackage{amssymb}
\usepackage{geometry}
\usepackage{tikz}
\usetikzlibrary{arrows}

\geometry{a4paper, total={170mm,257mm}, left=20mm, top=20mm}
\AtBeginEnvironment{align}{\setcounter{equation}{0}}
\AtBeginEnvironment{eqnarray}{\setcounter{equation}{0}}

\title{HW12 (CSCI-C241)}
\author{Lillie Donato}
\date{23 April 2024}

\begin{document}

\maketitle

\begin{enumerate}
    \item $\verb|first| = a$ \\ $\verb|rest| = \verb|[| b,c,d,e,f \verb|]|$
    \item $\verb|reverse|(L) = \begin{cases}
        \verb|[]| &\text{ if } L = \verb|[]| \\
        \verb|append|(\verb|reverse|(\verb|rest|), \verb|first|) &\text{ if } L = \verb|[first, *rest]|
    \end{cases}$
    \item Claim: $\verb|len|(\verb|reverse|(L)) = \verb|len|(L)$
    \begin{proof}
        \text{} \\
        (Base Step, $L = \verb|[]|$):
        \begin{eqnarray}
            \verb|len|(\verb|reverse|(\verb|[]|)) &=& \verb|len|(\verb|[]|) \\
            &=& 0 \\
            &=& \verb|len|(\verb|[]|)
        \end{eqnarray}
        (Inductive Step, $L = \verb|[first, *rest]|$): \\
        Assume for some list $\verb|rest|$, that $\verb|len|(\verb|reverse|(\verb|rest|)) = \verb|len|(\verb|rest|)$
        \begin{eqnarray}
            \verb|len|(\verb|reverse|(L)) &=& \verb|len|(\verb|reverse|(\verb|[first, *rest]|)) \\
            &=& \verb|len|(\verb|append|(\verb|reverse|(\verb|rest|), \verb|first|)) \hspace{1cm} \text{(By the definition of } \verb|reverse| \text{)}\\
            &=& \verb|len|(\verb|reverse|(\verb|rest|)) + \verb|len|(\verb|[first]|) \\
            &=& \verb|len|(\verb|reverse|(\verb|rest|)) + 1 \hspace{1cm} \text{(Because the length of a single item list is $1$)} \\
            &=& \verb|len|(\verb|rest|) + 1 \hspace{1cm} \text{(By the Induction Hypothesis)} \\
            &=& \verb|len|(\verb|[first, *rest]|) \hspace{1cm} \text{(By the definition of } \verb|len| \text{)}\\
            &=& \verb|len|(L)
        \end{eqnarray}
    \end{proof}
    \item Claim: Every \textbf{X-number} $n$ is divisble by $3$ ($\frac{n}{3} \in \mathbb{Z}$)
    \begin{proof}
        \text{} \\
        (Base Step, $n = 12$):
        \begin{eqnarray}
            \frac{12}{3} &=& 4 \\
            4 &\in& \mathbb{Z}
        \end{eqnarray}
        (Base Step, $n = 15$):
        \begin{eqnarray}
            \frac{15}{3} &=& 5 \\
            5 &\in& \mathbb{Z}
        \end{eqnarray}
        (Inductive Step, $n = x+y$, where $x,y$ are both \textbf{X-numbers}): \\
        Assume for some \textbf{X-numbers} $x,y$, that they are divisble by $3$
        \begin{eqnarray}
            x &=& 3a \hspace{1cm} \text{(By our Induction Hypothesis and where $a \in \mathbb{Z}$)} \\
            y &=& 3b \hspace{1cm} \text{(By our Induction Hypothesis and where $b \in \mathbb{Z}$)} \\
            x+y &=& 3a + 3b \hspace{1cm} \text{(Where $x,y$ are both \textbf{X-numbers})} \\
            &=& 3(a+b) \hspace{1cm} \text{($3(a+b)$ is divisible by $3$ because $3$ and $a+b \in \mathbb{Z}$, so $3(a+b) \in \mathbb{Z}$)}
        \end{eqnarray}
        (Inductive Step, $n = x-y$, where $x,y$ are both \textbf{X-numbers}): \\
        \begin{eqnarray}
            x &=& 3a \hspace{1cm} \text{(By our Induction Hypothesis and where $a \in \mathbb{Z}$)} \\
            y &=& 3b \hspace{1cm} \text{(By our Induction Hypothesis and where $b \in \mathbb{Z}$)} \\
            x-y &=& 3a - 3b \hspace{1cm} \text{(Where $x,y$ are both \textbf{X-numbers})} \\
            &=& 3(a-b) \hspace{1cm} \text{($3(a-b)$ is divisible by $3$ because $3$ and $a-b \in \mathbb{Z}$, so $3(a-b) \in \mathbb{Z}$)}
        \end{eqnarray}
    \end{proof}
    \item Claim: For any $n \in \mathbb{Z}$ where $n \geq 1$, $L_n = F_{n-1} + F_{n+1}$
    \begin{proof}
        \text{} \\
        (Base Step, $n = 1$):
        \begin{eqnarray}
            L_1 &=& 1 \hspace{1cm} \text{(By the definition of the Lucas Numbers)} \\
            &=& 1 + 0 \\
            &=& F_1 + F_0 \\
            &=& F_{2 - 1} + F_{2-2} \\
            &=& 0 + (F_{2 - 1} + F_{2-2}) \\
            &=& F_0 + F_2 \\
            &=& F_{1-1} + F_{1+1}
        \end{eqnarray}
        (Inductive Step): \\
        Assume for some $k \in \mathbb{Z}$ where $k \geq 1$, $L_i = F_{i - 1} + F_{i + 1}$ for all $1 \leq i \leq k$
        \begin{eqnarray}
            L_{k+1} &=& L_k + L_{k-1} \\
            &=& (F_{k-1} + F_{k+1}) + (F_{k-2} + F_k) \hspace{1cm} \text{(By the Inductive Hypothesis)} \\
            &=& (F_{k-1} + F_{k-2}) + (F_{k+1} + F_k) \\
            &=& F_k + F_{k+2} \\
            &=& F_{(k+1)-1} + F_{(k+1)+1}
        \end{eqnarray}
    \end{proof}
    \item Claim: Every $n \in \mathbb{Z}$ where $n \geq 1$, $n$ can be written as the sum of powers of $2$
    \begin{proof}
        \text{} \\
        (Base Step, $n = 1$):
        \begin{eqnarray}
            2^0 &=& 1
        \end{eqnarray}
        $1$ can be written in the sum of powers of $2$, because any number raised to $0$ is $1$ \\
        (Inductive Step): \\
        Assume for some $k \in \mathbb{Z}$ where $k \geq 1$, $i$ can be written as the sum of power of $2$, where $1 \leq i \leq k$ \\
        \begin{align}
            &\text{Case 1: } k \text{ is odd} \\
            &\hspace{1cm} \text{By our IH, } \frac{k+1}{2} \text{ can be written as a unique sum of powers of $2$,} \\
            &\hspace{1.5cm} \text{so } \frac{k+1}{2} = c_0 \cdot 2^0 + c_1 \cdot 2^1 + c_2 \cdot 2^2 + c_3 \cdot 2^3 + \dots + c_n \cdot 2^n \text{,} \nonumber \\
            &\hspace{1.5cm} \text{where all of the values multiplying the powers of $2$ are $0$ or $1$} \nonumber \\
            &\hspace{1cm} \text{Therefore, } \\
            &\hspace{1.5cm} k+1 = 2 \cdot \frac{k+1}{2} = c_0 \cdot 2^{0+1} + c_1 \cdot 2^{1+1} + c_2 \cdot 2^{2+1} + c_3 \cdot 2^{3+1} + \dots + c_n \cdot 2^{n+1} \nonumber \\
            &\text{Case 2: } k \text{ is even} \\
            &\hspace{1cm} \text{Since } k \text{ is even and can be represented in sums of powers of $2$ (according to our IH),} \\
            &\hspace{1.5cm} \text{then in the sum of powers of } 2 \text{ for } k \text{, } 2^0 \text{ is not a member} \nonumber \\
            &\hspace{1cm} \text{Therefore } k+1 = k + 2^0 \\
            &\text{In either case of } k+1 \text{, it can be made by a unique sum of powers of $2$, so in general it can be} \\
            &\hspace{.5cm} \text{made by a sum of powers of $2$} \nonumber
        \end{align}
    \end{proof}
\end{enumerate}

\end{document}